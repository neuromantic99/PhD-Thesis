\chapter{\label{ch:1-intro}Introduction} 
\minitoc

Key to the orchestration of behaviour by neural systems is that information, in the form of neural activity, is reliably and accurately transmitted between anatomically distinct brain regions performing specialised tasks. Activity is transformed at each stage of its journey, and circuits often perform multiple tasks in parallel \cite{mante_context-dependent_2013, stringer_spontaneous_2019}, so it is challenging to disentangle which facets of neural activity contribute to a specific behaviour or process. In-depth analysis of cellular resolution recordings of neural activity during sensory stimulation \cite{hubel_receptive_1962, simons_response_1978, carandini_linearity_1997, okun_diverse_2015, stringer_high-dimensional_2019}  and/or well-characterised behaviour \cite{britten_analysis_1992, platt_neural_1999, yang_probabilistic_2007, herzfeld_encoding_2015, lak_dopaminergic_2020} has begun to shed light on how the sensory world, decisions and actions are encoded in individual brain regions. However, how activity is structured in order to facilitate its journey through the brain is less well understood. It is especially critical that close attention is paid to unravelling how activity is propagated through the healthy brain, as dysfunction of this process is thought to play a key role in the aetiology of Alzheimer’s disease \cite{hazra_inhibitory_2013, busche_impairments_2016}, Parkinson’s disease\cite{mcgregor_circuit_2019} and epilepsy \cite{goldberg_mechanisms_2013}.

For researchers studying streams of sensory processing, it is advantageous to circumvent the transformations of neural activity occurring at regions upstream from the brain region under study. This can be achieved by taking experimental control of neural activity and driving spikes directly through stimulation of individual neurons \cite{houweling_behavioural_2008, tanke_single-cell_2018, chettih_single-neuron_2019} or groups of neurons \cite{romo_somatosensory_1998, huber_sparse_2008, histed_cortical_2014, dalgleish_how_2020, gill_precise_2020}. The recent renaissance in optical methods enables this to be achieved with unparalleled precision and flexibility. Specifically, two-photon optogenetics allows researchers to drive a predefined number of spikes in between one and hundreds of neurons with near single-cell precision, with the identities and number of neurons stimulated varied on a trial-by-trial basis. Neural responses can vary trial-by-trial to even identical stimulus presentations \cite{britten_responses_1993, faisal_noise_2008, softky_highly_1993}, rendering the flexibility endowed by two-photon optogenetics critical for tightly controlling the evoked response in the sensory stream and observing its effect on behaviour. Moreover, direct experimental stimulation of neurons is likely a critical tool for understanding activity propagation, as the activity of anatomically connected brain regions is typically highly correlated \cite{honey_predicting_2009, musall_single-trial_2019}, but it is often unclear whether this correlated activity results from propagation between regions or from common input. Direct stimulation circumvents this issue as activity locked temporally to stimulation of a connected region likely arises from propagation. Thus, we can understand how connected brain regions coordinate by directly stimulating a given brain region, and recording a connected region, during behaviour that is causally underpinned by propagating activity.

To approach this problem, my colleagues and I imaged two hierarchically organised, densely interconnected and functionally well-characterised \cite{pons_physiological_1987, kamatani_experience-dependent_2007, aronoff_long-range_2010, chen_behaviour-dependent_2013, yamashita_membrane_2013, chen_long-range_2016, kwon_sensory_2016, yamashita_target-specific_2016} regions, the primary and secondary somatosensory cortex (S1 and S2) while performing two-photon optogenetic photostimulation of S1 neurons. We assigned behavioural salience to photostimulation by training mice through operant conditioning to report detection of the stimulation. Photostimulation was targeted to varying groups of 5-150 neurons in S1, allowing us to parametrically adjust the effect of photostimulation, monitor its effect on behaviour and record trials spanning the perceptual threshold of the animal. By simultaneously recording neural activity occurring before and after photostimulation in both brain regions, we were able to assess how behaviourally salient stimulation is propagated through anatomically distinct brain regions.

We demonstrate that behaviourally salient photostimulation of S1 (hit trials) elicits robust activation of S2, which represents stimulus information for several seconds after stimulation. Further, we find that ongoing activity in S1 and S2, immediately preceding stimulation, influences the detectability of the upcoming stimulus. Specifically, the photostimulus detectability and propensity to propagate is greatest when both the population variance pre-stimulus in S1 is minimised and the number of cells stimulated is maximised. This is consistent with a signal-to-noise framework in which the signal is the number of photostimulated cells and the noise is the variance of the ongoing activity in S1. Thus we demonstrate that the signal-to-noise ratio of an input to a neural circuit is critical both for the behavioural salience of the input and the likelihood that the input is reliably propagated beyond the brain region in which it arises.

In this project I developed, alongside my colleagues, a novel preparation which allowed us to simultaneously read and write propagating neural activity in the mouse. This was achieved by combining single-cell resolution two-photon calcium imaging of neural activity propagating between cortical regions, and two-photon optogenetic stimulation of small groups of neurons which causally drove behaviour. To run these experiments, I built both hardware and software to run behavioural training and optogenetic stimulation in an automated manner, requiring minimal experimenter intervention during sessions and which automatically progressed mice through stages of training. During the final stages of training, optogenetic stimulation was delivered using holographic two-photon photostimulation, a cutting-edge optical method which has not previously been used in a preparation spanning multiple cortical regions. Finally, I performed transformations and in-depth statistical modelling of the resulting data to draw conclusions about cortical function.

In this thesis, I introduce the methods and biological theory that underpin the project, outline how these techniques are employed in our novel preparation, and explain how we employed statistical modelling to draw conclusions from the data. Finally I set these findings in the context of the field and explore how they might act to advance our understanding of the neural code.

\section{Recording neural activity}

Cortical computation is likely underpinned by populations of neurons  rather than individual neurons acting in isolation \cite{averbeck_neural_2006, kohn_correlations_2016, panzeri_cracking_2017}. Hence, to understand neural coding in cortex, it is necessary to record the activity of large groups of neurons simultaneously in vivo. These recordings were first carried out through electrical methods in which one or more microelectrodes are inserted into the extracellular space around neurons, allowing the action potential waveform to be recorded from multiple cells \cite{hubel_tungsten_1957, mcnaughton_stereotrode_1983, buzsaki_large-scale_2004}. Recent developments to this technique has resulted in electrodes with capable of hundreds of neurons simultaneously, through thousands of channels, with exquisite temporal resolution \cite{steinmetz_neuropixels_2021}. However, despite the power of this method, it has a number of drawbacks. It is often not possible to track the genetic identity or spatial relationship of recorded cells, analysis to parcel out recorded activity to individual neurons is complex \cite{harris_improving_2016}, and stimulation of neurons during electrical recordings is limited to single-cell stimulation \cite{margrie_vivo_2002, houweling_nanostimulation_2010} or simultaneous activation of groups of spatially co-localised neurons \cite{penfield_w_somatic_1937, asanuma_functional_1967, salzman_cortical_1990, romo_somatosensory_1998, cardin_driving_2009, kim_integration_2017}.

More recently, fluorescence microscopy has been applied to functional imaging in vivo to circumvent these issues and record large populations of neurons with single cell resolution. Generally, neural activity is recorded optically using calcium imaging \cite{grienberger_imaging_2012}, in which neurons are genetically engineered to express fluorescent molecules which undergo a conformational change when bound to calcium. As a result, the fluorescence intensity of neurons expressing these molecules increases following calcium influx driven by an action potential \cite{berridge_neuronal_1998} and spikes can be recorded optically. Currently, the most popular way to perform calcium imaging in vivo is through genetically encoded calcium indicators \cite{looger_genetically_2012} such as GCaMP \cite{chen_ultrasensitive_2013, dana_high-performance_2019}. These can be expressed in neurons long-term through viral injection \cite{packer_simultaneous_2015} or in transgenic animals \cite{huang_relationship_2021} enabling chronic imaging of the same neurons over time \cite{andermann_chronic_2013}. While calcium imaging addresses the drawbacks of electrophysiology by visualising neurons, the time course of the indicator can span hundreds of milliseconds, meaning it has sub-standard temporal resolution compared to electrophysiology and resolving single-spike events can be challenging \cite{huang_relationship_2021}. Most germane to this thesis however, calcium imaging is the only method of recording neural populations that has thus far been combined with targeted two-photon photostimulation \cite{packer_simultaneous_2015} (see below).

Fluorescence microscopy \cite{lichtman_fluorescence_2005} requires a light-source to excite the fluorophores in the sample. This usually takes the form of a LED or laser which bathes the entire sample simultaneously in light, allowing a sensor, such as a camera, to detect emitted photons and create an image of the sample. This is known as ‘widefield’ or ‘one-photon’ excitation However, biological tissue is highly scattering, meaning a photon emitted from a given location on the sample may not arrive at the exact corresponding location on the sensor; additionally illumination causes the specimen to fluorescence throughout the z-dimension, resulting in photons emitted beyond the desired focal plane. Taken together these phenomena degrade the spatial resolution of the image. By contrast, scanning microscopy creates ‘optical sectioning’ by sampling regions of the sample in sequence, meaning detected photons can be assigned to a known location of the specimen \cite{lichtman_fluorescence_2005}. While confocal microscopy \cite{nwaneshiudu_introduction_2012} is the most popular form of scanning microscopy it is of little use to in vivo mammalian neuroscience, as the wavelengths of light used to excite common fluorophores do not penetrate beyond the most superficial layers of the brain \cite{helmchen_deep_2005}. This issue is circumvented by two-photon microscopy which, usually, employs near-infrared wavelength lasers producing photons of roughly half the energy required to excite the fluorophore in the sample. These long-wavelength lasers penetrate deep into tissue with minimal scattering and generate optical sectioning of the specimen through multiphoton absorption. Multiphoton absorption occurs when two photons arrive simultaneously at the fluorophore and excite it by combining their energies, creating optical sectioning as the probability of this event occurring is vanishingly small beyond the focal point of the excitation beam. Hence detected photons can be confidently assigned to a small point on the sample which produces a high-resolution image up to several hundred microns deep in a sample of neural tissue \cite{helmchen_deep_2005}.

Thus by combining calcium imaging with two-photon microscopy, the activity of hundreds of neurons can be recorded in awake behaving mice with minimal tissue lesioning \cite{helmchen_vivo_1999, stosiek_vivo_2003, lutcke_steady_2013, driscoll_dynamic_2017, stringer_spontaneous_2019}.

\section{Manipulating neural activity with light}

Experiments which rely purely on recording neural activity during behaviour can be very challenging to design and interpret as it is unclear which of the sprawling array of brain regions, cell types, projection targets and neural features of interest are driving the behaviour without artificial deletion or manipulation. The current leading method in the neuroscientist’s toolkit to address this is optogenetics, which combines optical and genetic technology to allow experimenters to take direct control of neurons and neural circuits. Briefly and generally, optogenetic experiments are conducted by engineering a light-sensitive protein (opsin) to a neuron, or other cell type. This allows the experimenter to shine light onto the cell and drive a current which can, for example, increase or decrease a neuron's firing rate (for reviews see \cite{miesenbock_optogenetic_2009, deisseroth_optogenetics_2011}). Optogenetics has unparalleled versatility in that experimenters can deliver loss or gain of function restricted to individual brain regions \cite{zhang_optogenetic_2010}, genetic subtypes \cite{cardin_driving_2009, tye_dopamine_2013, zalocusky_nucleus_2016} or projection targets \cite{mattis_frequency-dependent_2014} in a behaving animal \cite{tye_amygdala_2011} with tight temporal control \cite{krook-magnuson_-demand_2013}. 

However, in a similar vein to the issues raised above with fluorescence imaging, using widefield illumination to drive optogenetic stimulation is spatially imprecise and results in coactivation of hundreds to thousands of neurons simultaneously \cite{huber_sparse_2008}. This likely evokes non-physiological patterns of activity, conflicting with the sparse coding regime that has been observed in cortex \cite{olshausen_sparse_2004}. It is also unlikely that all stimulated neurons can be recorded. Additionally, widefield optogenetics can be targeted only to genetic subtypes of neurons and not to functionally-defined subtypes, which may drive a behaviour but be distributed throughout one or more brain regions.  Finally, the same neurons are likely stimulated on each trial, meaning the effect of stimulation is challenging to vary on a trial-by-trial basis.

Recently, researchers expanded the use of two-photon technology beyond recording neurons to controlling them through optogenetics (Rickgauer, Deisseroth and Tank, 2014; Packer et al., 2015; Hernandez et al., 2016; Mardinly et al., 2018; Chettih and Harvey, 2019; Marshel et al., 2019; Daie, Svoboda and Druckmann, 2021). This greatly improves the spatial resolution of the stimulation and hence addresses the issues outlined above. The optical sectioning afforded by a two photon laser beam means that stimulation can be restricted to individual neurons. Additionally the targeted neuron(s) can be visually identified by the experimenter, meaning that, when combined with calcium imaging, functional neuronal subtypes can be identified which, for example, exhibit correlated firing with a specific behavioural event. Thus by photostimulting such neurons, their causal role in the behaviour can be assessed (Chettih and Harvey, 2019; Marshel et al., 2019; Daie, Svoboda and Druckmann, 2021).

While standard two-photon beam paths can be used to target single neurons for photostimulation, targeting multiple neurons requires more specialised optics to split the beam. The most widely used way of achieving this is through digital holography, in which a spatial light modulator (SLM) 	is incorporated into the beam path. Through a reflective liquid-crystal layer, this acts as a programmable two-dimensional diffraction grating, allowing arbitrary patterns to be created from the two-photon beam (Lutz et al., 2008). Thus, the researcher is able to target arbitrary combinations of neurons for optogenetic manipulation changing the identify, combination or quantity of targets on a trial-by-trial basis. 

Recently, the toolkit of opsins available to researchers has expanded to include proteins particularly suited to \cite{yizhar_neocortical_2011} or specifically designed for \cite{chettih_single-neuron_2019, mardinly_precise_2018, marshel_cortical_2019} two-photon optogenetic experiments. Two specific advances utilised in this thesis is the development of opsins with a red-shifted excitation spectrum, and opsins with somatic-targeting motifs. Red-shifted opsins are useful as they can be used in simultaneously with fluoresence imaging of proteins excited by blue light. Somatic-targeting improves the resolution of photostimulation, as a beam of light targeted to a cell soma is less likely to cause off-target stimulation by activating the neurites of other cells overlaying the targeted soma.

\section{All-optical interrogation of neural circuits in vivo}
 
The two-photon techniques outlined above, calcium imaging and optogenetics can be combined \textit{in vivo} to provide a unique window into neural circuit function, enabling researchers to read and write neural activity on the level of individual neurons in behaving animals. This “all-optical” approach has yielded insight into a variety of mammalian neural systems, a non-exhaustive list of which is reviewed here. The first time all-optical interrogation was applied to in vivo circuit neuroscience was in the hippocampus to stimulate place cells while mice navigated in a virtual environment(Rickgauer, Deisseroth and Tank, 2014). The authors demonstrated that two-photon photostimulation could be used to mimic activity observed in its natural place field. The study also revealed that individual place cells modulate the firing of neighbouring place cells to a greater extent than previously thought, the previous model(Andersen, Bliss and Skrede, 1971) stating that “small strips of the hippocampal cortex may operate as independent functional units”. The idea of using all-optical methods to study the influence of cortical neurons on other neurons in the local circuit was subsequently employed in the visual cortex(Chettih and Harvey, 2019). Activation of excitatory neurons elicited a center-surround spatial influence profile, in which neurons close to the stimulated cell (< 70 μm) were on average excited, whereas neurons proximal but not directly neighbouring (70 - 300 μm) were on average inhibited. Influence was balanced at distances > 300 μm. Interestingly after mapping the visual tuning of neurons, the authors found that single-neuron photostimulation was more likely to drive suppression of another neuron if the two neurons exhibited similar tuning, hinting that visual cortex may encoding stimuli through feature-competition which enhances visual acuity, sparsifies the neural representation and reduces redundancy.  All-optical techniques have also been used to manipulate active behaviour directly. In a technical tour-de-force in which they also developed new optics and an improved opsin specifically for all-optical experiments, Marshel et al.(Marshel et al., 2019) performed holographic two-photon photostimulation of ensembles of neurons throughout the cortical column of the primary visual cortex (V1). They showed that photostimulating co-tuned ensembles in trained mice could result in the same behavioural outcome as visual stimulation. Further, photostimulation of a subset of the neurons in an ensemble resulted in pattern completion of the remainder on the ensemble, hinting that cortical neurons exist in functional subnetworks. Finally, in anterolateral motor cortex, two-photon photostimulation of sparse subsets of neurons during behaviour elicited persistent neural activity and biased behaviour for several seconds following stimulation. This is indicative of recurrent connectivity in cortical circuits that acts to maintain inputs that are relevant to behaviour. In sum, combining two-photon calcium imaging and optogenetics has enhanced our understanding of cortical computation and its relationship to behaviour. 

\section{Somatosensory cortex as a model system}

For nocturnal animals living in tunnels and burrows, somatosensation is a sense critical for survival. In rodents, the whiskers form by far the largest single somatotopic representation as these organs are crucial for navigation, threat and food detection and object localisation(Gustafson and Felbain-Keramidas, 1977). These properties make the rodent whiskers and their somatotopic cortical representation, the ‘barrel cortex’, an ideal model system for the study of mammalian sensory processing. This is because the barrel cortex is easily accessible on the cortical surface, the sensory input is highly salient to the animal and the whiskers are easily manipulated to generate stimuli or drive an experimenter controlled behaviour. Indeed, mice have been shown to exhibit exquisite sensitivity in their ability to discriminate textures using their whiskers(Wu et al., 2013).  Finally, barrel cortex activity operates in a sparse coding regime(Crochet et al., 2011) as a result, targeted optogenetic stimulation of small groups of neurons is more likely to impact behaviour than in a less sparse area.

Due to its suitability as a model system, the physiology(Feldmeyer et al., 2013) and anatomy(Petersen, 2007) of barrel cortex is well characterised and extensively studied. Generally, stimuli resulting from deflection of a whisker is represented as depolarisation throughout the barrel corresponding barrel(Crochet et al., 2011) followed by a prolonged period of inhibition(Simons, 1978) and spreading depolarisation to neighbouring barrels(Ferezou, Bolea and Petersen, 2006). The features of sensory stimulation that barrel cortical neurons are tuned to is not fully understood and is likely more complex than primary visual cortex. Neurons have been proposed to a variety of sensory events, including, but not limited to: the product of the frequency and amplitude of a whisker deflection(Arabzadeh, Petersen and Diamond, 2003), stick/slip events(Jadhav, Wolfe and Feldman, 2009), texture coarseness(Garion et al., 2014) and object location(O’Connor et al., 2010).  In sum, primary sensory neurons in the somatosensory system exhibit responses spanning a variety of axes in stimulus space, making it a rich target for interrogating mammalian sensory cortical computation. 

\section{Primary and secondary somatosensory cortex}

Although it is by far the most extensively studied somatosensory region, S1 does not operate in isolation, forming hierarchically organised recurrent connections with higher somatosensory regions(Felleman and Van Essen, 1991; Kamatani et al., 2007; Kwon et al., 2016). One of the major outputs of S1 is S2 (the secondary somatosensory cortex), from which it also receives feedback connections, and activity relating to sensory processing has been shown to flow readily between these regions(Aronoff et al., 2010). Additionally communication between S1 and S2 is thought to be crucial for behaviour. These two regions have been shown to show enhanced coordination during active behaviours such as goal directed licking, with directly projecting neurons increasing their activity during behaviourally relevant touch(Chen et al., 2016). Further, S2 projecting neurons in S1 show whisking related activity not observed in primary motor cortex (M1) projecting neurons, and S2 projecting neurons encoded texture discrimination with greater fidelity than those projecting to M1(J. L. Chen et al., 2013). Finally, during a perceptual task, feed-forward neurons projecting from S1 to S2 have been shown to carry information relating to choice, with the S1-S2 loop generally encoding information related to both active touch and to behavioral choice(Kwon et al., 2016). Taken together, these results show that communication between S1 and S2 in the rodent is likely critical to paint a picture of the sensory world and guide decision making.  

\section{Propagating neural activity}

As discussed above, brain areas must interact and share information to process stimuli and generate behaviour, yet how neurons shape their activity to mediate interactions is not well understood. Previous studies have studied this phenomenon by recording from multiple brain regions simultaneously during sensory stimulation and/or active behaviour. Normally, interactions are assessed by correlating activity between the two regions on a single cell or population level. Such studies have posited that: color perception may rely on communication between color responsive V1 and V2 neurons(Roe and Ts’o, 1999), attention increases coordination between visual cortical regions(Ruff and Cohen, 2016), motor preparatory activity is formatted so it is not prematurely propagated(Kaufman et al., 2014) and that activity forming interactions between brain regions exist in a communication subspace(Semedo et al., 2019). 
Communication between subnetworks has also been assessed in silico, which has further developed our understanding of how neural activity is processed to facilitate propagation.. These studies have suggested that propagation is underpinned by balanced amplification, in which excitation generated by long-range recurrent connectivity is balanced by local feedback inhibition, which stabilises the network(Joglekar et al., 2018). Further, in feed-forward network models, intrinsic noise in the firing rates of neurons is critical for feed-forward propagation of inputs(Vogels and Abbott, 2005; Ozer et al., 2010). Finally the covariance structure of local population activity is thought to be critical for the propagation of activity downstream. This is because there exists a family of covariance structures that optimally propagate activity through noisy circuits; these covariances are distinct from those that represent inputs with the highest fidelity(Zylberberg et al., 2017). As the brain must both represent stimuli and propagate information it is possible that the covariance structures that exist in vivo may exist as a trade-off between these two phenomena, though this has not yet been assessed.  

In silico studies and recordings made from multiple brain regions simultaneously have provided crucial insight into how neural activity is formatted to facilitate propagation through regions of the brain. However, causal interventions are required in vivo to give biological backing to in silico work and ensure that activity recorded in vivo is being propagated between the regions of interest, and does not result simply from common afferent input. Moreover casual manipulations can tie studies into activity propagation to behaviour by ensuring that the recorded propagating activity is behaviourally relevant. 

\section{Driving behaviour directly through cortical stimulation}

The most direct way to link recorded neural activity to behaviour is by developing a behavioural paradigm dependent by definition on the recorded neurons. This is achieved by training animals to make an action based solely on experimental manipulation of activity, thus greatly simplifying the interpretation of the link between recorded activity and behavioural outcome. In primates for example, this method has been used to establish a causal link between the probability of somatosensory cortical neurons firing and the frequency of a stimulus applied to the fingertip(Romo et al., 1998). This was achieved by training monkeys to make an action based on two frequencies of fingertip stimulation. Animals still performed the task accurately after replacing one of the stimuli with the same frequency electrical stimulation of cortical neurons in the somatotopic region corresponding to the fingertip. This ‘illusory percept’ thus demonstrates that activity in these neurons (and those activated downstream) is sufficient to guide behaviour. Surprisingly this same concept has also been applied to face recognition(Afraz, Kiani and Esteky, 2006), through a task in which monkeys were trained to distinguish face and non-face visual stimuli. This study linked face-responsive neurons in the inferior temporal cortex causally to face perception by showing that electrical stimulation of small groups of these neurons drove face detection. Thus, through causal intervention the authors unequivocally linked neurons correlative with facial stimuli to face to the active behaviour of face perception.

The idea of driving behaviour directly through electrical stimulation has also been applied in rodents. Most notably, Houweling and Brecht (2008)(Houweling and Brecht, 2008) showed that nanostimulation of single barrel cortical neurons drove behaviour to slightly above chance level, indicating that rodents are able to detect spikes in a single neuron. While this hints that the individual neurons may be more important in sensory processing than standard population coding frameworks(Averbeck, Latham and Pouget, 2006) suggest, as behavioural performance is only slightly statistically above chance, it is unlikely that single neurons underpin ethological behaviours.

Behaviour can also be driven directly through optical stimulation. This was first demonstrated by Huber et al. (2008)(Huber et al., 2008) who trained freely moving mice to lick a port to receive a reward following widefield optogenetic stimulation of S1. From this they attempted to elucidate how many neurons were required to spike in order to drive behaviour robustly (rather than to just above chance level as in Houweling and Brecht). The authors indirectly estimated that mice could detect trains of action potentials when ~60 neurons were activated simultaneously. This idea was expanded upon through the use of two-photon holographic photostimulation which has the advantage of being able to target directly a specific number of neurons, with the number varying on a trial-by-trial basis. Through this, it was shown that trains of action potentials in only 30-40 S1 neurons are required to robustly drive licking behaviour(Dalgleish et al., 2020); it is unclear whether more or less neurons are required to be stimulated in other brain regions to elicit behaviour. 

Further, direct optical driving of behaviour has given insight into its computational underpinnings beyond that of the minimum number of neurons required. Through LED stimulation of S1, Histed and Maunsell (2014)(Histed and Maunsell, 2014) showed that only the total number of injected spikes, across a 100 ms window, influenced behavioural performance, the frequency or length of stimulation had no effect on performance. Hence the authors posit that animals detect stimuli by linear integration of the total number of spikes in a 100 ms window. This implies that spike timing has no impact on cortical computation and behaviour, at least in this highly non-ethological condition. However an apparently conflicting study(Gill et al., 2020) employing two-photon holographic photostimulation hints that spike counting may not be the complete picture. The authors trained mice to detect targeted activation of small groups of neurons in the olfactory bulb. They found that synchronous (less than 10 ms jitter) activation of < 30 neurons was detected more readily than asynchronous activation ($\geqslant$ 30 ms jitter). This discrepancy between the two studies may be explained as first, they were conducted in different brain regions, and second two-photon holographic activation of a small number of neurons is likely much closer to the noise threshold compared to widefield activation of a large number of neurons. Hence synchronous activation may be critical if the stimulus is weak, whereas only spike counting may be required for stronger stimuli.

You might need a section on relevant neural features like E/I balance, maybe you could just pad out the bit about how neural activity is formatted

\section{Summary}

This thesis aims to stitch together the ideas presented above by applying behaviour driven directly by holographic two-photon photostimulation to the study of activity propagation between cortical regions. The barrel fields of the primary and secondary somatosensory cortex form the ideal model system for this study as they are easily optically accessible,  known to be densely connected with this connectivity likely critical for behaviour, and it is known that experimental manipulation of S1 neurons can directly drive behaviour. Through this preparation I will examine in vivo how neural activity is formatted in S1 to enable robust propagation of activity through the cortical hierarchy, resulting in the communication between brain regions that is critical for neural processing.
